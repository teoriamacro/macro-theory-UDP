\documentclass{article}

\usepackage[utf8]{inputenc}
\usepackage[T1]{fontenc}
\usepackage{geometry}
\geometry{a4paper}

\usepackage[spanish]{babel}

\title{INDIVISIBLE LABOR AND BUSSINES CYCLE}
\author{Gary D. Hansen}
\date{1985}

\begin{document}
\maketitle

Ficha del texto de Gary D. Hansen

\section{Información de cita}


Título: Indivisible Labor and Bussines Cycles


Autor: Gary D. Hansen

Fecha: 1985

Publicación:Journal of Monetary Economics

Origen: University of California, Santa Barbara, Estados Unidos

\section{Idea Principal}

La idea principal consiste en un modelo de  crecimiento económico con shocks tecnológicos.

El trabajo es indivisible por lo que toda la variabilidad de las horas trabajadas se debe a fluctuaciones en el número de trabajadores. Es diferente a los moedelos anteriores de equilibrio de ciclos económicos, esta economía presenta grandes variaciones en las horas de trabajo. En este modelo se estudia con división del trabajo y se toman datos de series de tiempo para el periodo post guerra. 


\section{Secondary Claims}

En relación al trabajo indivisible, la hipótesis planteada tiene motivación en la observación en que gran parte de las personas prefiere trabajar a tiempo completo o trabajar nada.




En el articulo se demuestra  que en una economía que tiene exposiciones laborales indivisibles con grandes variaciones en el número de horas de trabajo en relación a la productividad.

\section{Lista de Conceptos}

 No convexidad

Estocástico

Trabajo indivisible


Fluctuación económica agregada


Ciclo económico


Elasticidad de sustitución

Productividad

\section{Trabajos más citados}

Lucas y Prescott (1971)

Lucas (1977)

Lucas y Prescott (1984)

Kydland y Prescott (1982)

Kydland y Prescott (1984)

Rogerson Indivisible labour, lotteries and equilibrium, Unpublished manuscript (University of Rochester, Rochester, NY)

\section{Limitación}

La elasticidad de sustitución del ocio en dieferentes periodos para toda la economía es infinita e independiente de la elasticidad  de sustitución de la función de utilidad de los individuos.

Un modelo con trabajo divisible no presenta grandes variaciones en número de horas trabajadas con respecto a la productividad

Las convexidades no hacen posible un modelo de equilibrio del ciclo económico para exhibir variaciones del empleo. Por lo tanto, las no convexidades afectan considerablemente el equilibrio futuro.


\end{document}





