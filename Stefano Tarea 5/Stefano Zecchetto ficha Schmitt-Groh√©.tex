\documentclass{article}

\usepackage[utf8]{inputenc}
\usepackage[T1]{fontenc}
\usepackage{geometry}
\geometry{a4paper}

\usepackage[spanish]{babel}

\title{BUSSINES CYCLES WITH A COMMON TREND IN NEUTRAL AND INVESTMENT-SPECIFIC PRODUCTIVITY}
\author{Stephanie Schimtt-Grohé, Martín Uribe}
\date{3 de Agosto de 2010}

\begin{document}
\maketitle

\section{Información de cita}

Autor: Stephanie Schmitt-Grohé, Martín Uribe

Fecha: 3 de agosto de 2010

Publicación: Review of Economic Dynamics

Origen:  Department of Economics, Columbia University

\section{Idea Principal del artículo}

El articulo presente logra identificar una nueva manera de variación de los RBC (Real Bussines Cycle). 

Es muy importante la consideración de una tendencia estocástica común  entre la TFP (Total factor prodctivity) y el precio relativo de la inversión. Empiricamente se investigan dos temas: Uno tiene que ver con que si la TFP y el precio relativo de la inversión tienen tendencia estocástica y segundo es si estas dos series están cointegradas.

En el articulo se menciona también una gran contribución, la cual es la cuantificación de la importancia de los shocks a la tendencia estocástica común entre TFP y precio relativo de la inversión, la cual condujo las fluctuaciones agregadas en el periodo de la post guerra con datos de los Estados Unidos. 

En el modelo, se encuentra las innovaciones en la tendencia estocástica común que explican una parte considerable de las variaciones incondicionales de la producción, el consumo, la inversión y horas trabajadas.
 
\section{Secondary Claims}

En el modelo, para la estimación, un objetivo es comprobar la contribución a las fluctuaciones de los nuevos shocks ha saber, la tendencia estocástica comun en productividad neutral y de inversión específica.

Establecer cuales son los supuestos sobre shocks subyacentes los cuales entregarían el patrón de cointegración entre TFP y el precio relativo de la inversión.

\section{Lista de Conceptos}

Cointegración

Tendencia estocástica estacionaria

TFP

Precio relativo de la inversión

Shock comunes

Fuentes de ciclos económicos

Shock de tecnología

Shocks específicos de inversión

Ciclo económico

\section{Trabajos más citados}

Kydland y Prescott (1982)

King (1988)

Cogley y Nason (1995)

Rotenberg y Woodford (1996)

Greenwood (2000)

Fisher (2006)

Smets y Wouters (2007)

Justiniano (2008)

\section{Limitación}

La hipótesis que la TFP y el precio relativo de la inversión son impulsados por una tendencia estocástica comun no puede ser rechazada de mano. Dada la evidencia empírica, no es la única caracterización émpirica viable para el comportamiento de largo plazo de la TFP y el precio relativo de la inversión. 

\end{document}